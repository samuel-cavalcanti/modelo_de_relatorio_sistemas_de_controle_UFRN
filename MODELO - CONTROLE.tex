% Relatório da versão 1 do software ipump para o curso
% Sistemas de Controle - DCA0206 - UFRN
% Autores:
%   AUGUSTO MATHEUS PINHEIRO DAMASCENO
%   MARCEL DA CÂMARA RIBEIRO DANTAS
%   PABLO HOLANDA CARDOSO
%   PEDRO DE CASTRO GURGEL LIMA
%   RODRIGO DANTAS DA SILVA
% Modificado por: Ícaro Bezerra Queiroz de Araújo, Samuel Cavalcanti
%
% bugs e sugestões devem ser enviados como issues no repositório abaixo
% https://github.com/samuel-cavalcanti/modelo_de_relatorio_sistemas_de_controle_UFRN

%%%%%%%%%%%% STRUCTURE %%%%%%%%%%%%%%%
\documentclass[a4paper,12pt]{article}
\usepackage[T1]{fontenc}
\usepackage[utf8]{inputenc}
\usepackage[brazil]{babel}
\usepackage{lmodern}
\usepackage{setspace}
\usepackage[top=2cm, bottom=2cm, left=2cm, right=2cm]{geometry}
%%%%%%%%%%%%%%%%%%%%%%%%%%%%%%%%%%%%%%

%%%%%%%%%%%%%%%% PAGES STYLE %%%%%%%%%
\usepackage{fancyhdr}
\fancypagestyle{main}{
\renewcommand{\headrulewidth}{0pt}
\fancyhead[RO]{\thepage}
\fancyfoot[CO]{}
}
%%%%%%%%%%%%%%%%%%%%%%%%%%%%%%%%%%%%%%

\usepackage{graphicx}
\usepackage{epstopdf}
\usepackage{subfig}
\usepackage{mathptmx}
\usepackage{changepage}
\usepackage[abbr]{lib/harvard}

%\usepackage[alf]{abntex2cite}

%%%%%%%%%%% PDF METADATA %%%%%%%%%%%%%
\usepackage[ pdftitle={MODELO RELATÓRIO},
pdfsubject={INTRODUÇÃO AO LABORATÓRIO DE CONTROLE - Grupo 3},
pdfkeywords={Controle,Automação,UFRN,DCA,ipump},
hidelinks]{hyperref}
%%%%%%%%%%%%%%%%%%%%%%%%%%%%%%%%%%%%%%

\newcommand{\cityandyear}{
\large Natal-RN\\
 \the\year 
 }

\begin{document}




\onehalfspacing

\thispagestyle{empty}

\setcounter{page}{1}

%%%%%%%%%%%% LOGOS %%%%%%%%%%%%%%%%%%%

\begin{figure}[!ht]

\centering

\subfloat{
\includegraphics[width=2.7cm]{UFRN.pdf}
\label{UFRN Logo}
}
\hspace{11.09cm}
\subfloat{
\includegraphics[width=2.4cm]{DCA.pdf}
\label{DCA Logo}
}

%\caption{}
\label{Logos}

\end{figure}

%%%%%%%%%%%%%%% CAPA %%%%%%%%%%%%%%%%%

\vspace{-1cm}

\begin{center}
{\bf{\normalsize UNIVERSIDADE FEDERAL DO RIO GRANDE DO NORTE\\
CENTRO DE TECNOLOGIA\\
DEPARTAMENTO DE ENGENHARIA DE COMPUTAÇÃO E AUTOMAÇÃO\\
CURSO DE ENGENHARIA DE COMPUTAÇÃO
}}


\vspace{3.6cm}

{\bf{\large RELATÓRIO DA Nº EXPERIÊNCIA\\
TÍTULO DA EXPERIÊNCIA\\
}}
\vspace{1.5cm}
{\large TURMA: \\
	GRUPO Nº}

\vspace{3.6cm}



\begin{flushright}
\begin{normalsize}
NOME COMPLETO 1º ALUNO: Nº MATRÍCULA\\
\vspace{0.8cm}
NOME COMPLETO 2º ALUNO: Nº MATRÍCULA\\
\vspace{0.8cm}
NOME COMPLETO 3º ALUNO: Nº MATRÍCULA\\
\vspace{0.8cm}
NOME COMPLETO 4º ALUNO: Nº MATRÍCULA\\
\end{normalsize}
\end{flushright}


\vspace{2.5cm}

\cityandyear

\end{center}

\newpage

%%%%%%%%%%%%%%%  CONTRA-CAPA %%%%%%%%%

\thispagestyle{empty}

\begin{center}
\begin{normalsize}
NOME COMPLETO 1º ALUNO: Nº MATRÍCULA\\
\vspace{0.8cm}
NOME COMPLETO 2º ALUNO: Nº MATRÍCULA\\
\vspace{0.8cm}
NOME COMPLETO 3º ALUNO: Nº MATRÍCULA\\
\vspace{0.8cm}
NOME COMPLETO 4º ALUNO: Nº MATRÍCULA\\

\end{normalsize}
\end{center}
\vspace{3cm}

{\bf{\large {\centering TÍTULO DA EXPERIÊNCIA\\}}}

\vspace{4cm}

\begin{adjustwidth}{7.5cm}{0cm}

{\normalsize

Primeiro Relatório Parcial apresentado à disciplina de
Laboratório de Sistemas de Controle, correspondente à
avaliação da 1º unidade do semestre 2016.1 do 7º período
do curso de Engenharia de Computação e Automação da
Universidade Federal do Rio Grande do Norte, sob
orientação do {\bf Prof. Fábio Meneghetti Ugulino de
Araújo.}

}

\end{adjustwidth}

\vspace{2cm}

\begin{center}

Professor:  Fábio Meneghetti Ugulino de Araújo.

\vspace{2.5cm}

\cityandyear

\end{center}

\newpage

%%%%%%%%%%%%%%%  RESUMO %%%%%%%%%%%%%%

\thispagestyle{empty}

\begin{center}
{\large \textbf{RESUMO}}
\end{center}

\vspace{3cm}

\begin{flushleft}

\hspace{4ex}Trata-se da apresentação fiel, breve e concisa dos aspectos mais relevantes do
trabalho, apresentando as ideias essenciais, na mesma progressão e no mesmo
encadeamento que aparecem no texto. O resumo deve apresentar os objetivos, uma
visão geral, ampla e, ao mesmo tempo, clara e objetiva do conteúdo do trabalho.\\
\hspace{4ex}A norma da ABNT recomenda que se use de 150 a 500 palavras, em espaço
simples, e deve-se usar o verbo na voz ativa e na terceira pessoa singular. Logo abaixo,
devem ser colocadas as palavras-chave.\\

\end{flushleft}

\vspace{1.5cm}

\textbf{Palavras-chave:}

\newpage

%%%%%%%%%%%%%%%  LISTA DE SÍMBOLOS %%%

\thispagestyle{empty}

\begin{center}
{\large \textbf{LISTA DE SÍMBOLOS}}
\end{center}

\vspace{3cm}

\begin{tabular}{ l l }
A\hspace{1.5cm} & Matriz triangular superior com diagonal unitária.\\
\phantom{a} & \phantom{a}\\
D\hspace{1.5cm} & Matriz diagonal obtida a partir de $W^{T}W$\\
\phantom{a} & \phantom{a}\\
$\theta$\hspace{1.5cm} & Vetor de parâmetros.\\
\phantom{a} & \phantom{a}\\
$\Xi$\hspace{1.5cm} & Vetor de resíduos de modelagem.\\
\phantom{a} & \phantom{a}\\
d\hspace{1.5cm} & Tempo de retardo de um sistema ou tempo morto.\\
\phantom{a} & \phantom{a}\\
e(k)\hspace{1.5cm} & Resíduo (Erro de Estimação mais o Ruído).\\
\end{tabular}

\newpage

%%%% LISTA DE ABREVIATURAS E SIGLAS %%

\thispagestyle{empty}

\begin{center}
{\large \textbf{LISTA DE ABREVIATURAS E SIGLAS}}
\end{center}

\vspace{3cm}

\begin{tabular}{ l l }
ARX\hspace{1.5cm} & Matriz triangular superior com diagonal unitária.\\
ARMAX\hspace{1.5cm} & Matriz diagonal obtida a partir de $W^{T}W$\\
NARX\hspace{1.5cm}&Vetor de parâmetros.\\
NARMAX\hspace{1.5cm}&Vetor de resíduos de modelagem.\\
MQ\hspace{1.5cm}&Tempo de retardo de um sistema ou tempo morto.\\
\end{tabular}

\newpage

%%%%%%%%% LISTA DE FIGURAS %%%%%%%%%%%

\thispagestyle{empty}

\begin{center}
\listoffigures
\end{center}

\newpage

%%%%%%%%%%%%%%% SUMÁRIO %%%%%%%%%%%%%%

\thispagestyle{empty}

\begin{center}
\tableofcontents
\end{center}

\newpage

%%%%%%%%%%%%%%% INTRODUÇÃO %%%%%%%%%%%

\thispagestyle{main}

\section{INTRODUÇÃO}

\begin{flushleft}
\hspace{4ex}A introdução serve para o leitor ter uma noção genérica do tema que será
abordado. Uma boa introdução deve criar uma expectativa positiva no leitor e despertar
seu interesse pela leitura do restante do trabalho. Deve apresentar, basicamente, a
delimitação do assunto o(s) objetivo(s) do estudo e sua finalidade, o ponto-de-vista sob
qual o assunto será tratado, enfim, os elementos necessários para situar o tema do
trabalho.
\end{flushleft}

\newpage

%%%%%%%%%% REFERENCIAL TEÓRICO %%%%%%%

\thispagestyle{main}

\section{REFERENCIAL TEÓRICO}



Trata-se da apresentação do embasamento teórico sobre o qual se fundamentará
o trabalho, ou seja, são os pressupostos que darão suporte à abordagem do trabalho.
Lembrar de sempre que utilizar texto de outros lugares, utilizar citação da fonte, como por exemplo, \cite{LATEX04}.

\subsection{Seções}

\subsubsection{Subseções}

\newpage

%%%%%%%%%% METODOLOGIA %%%%%%%%%%%%%%%

\thispagestyle{main}

\section{METODOLOGIA}


\hspace{4ex}A metodologia é caracterizada pela explicação minuciosa dos procedimentos
técnicos realizados durante todo o trabalho.

\subsection{Seções}

\subsubsection{Subseções}

\newpage

%%%%%%%%%% RESULTADOS %%%%%%%%%%%%%%%

\thispagestyle{main}

\section{RESULTADOS}


\hspace{4ex}Neste capítulo, são apresentados e descritos os resultados obtidos dos experimentos feitos em laboratório. É importante que todos os gráficos e figuras apresentados neste capítulo estejam bem visíveis e com qualidade boa. Este é o capítulo mais importante do trabalho, pois é nele que o aluno irá descrever todos os resultados e observações obtidos no experimento.

\subsection{Seções}

\subsubsection{Subseções}

\newpage

%%%%%%%%%% CONCLUSÃO %%%%%%%%%%%%%%%

\thispagestyle{main}

\section{CONCLUSÃO}


\hspace{4ex}A conclusão, além de guardar uma proporção relativa ao tamanho do trabalho,
deve guardar uma proporcionalidade também quanto ao conteúdo. Não deve conter
assuntos desnecessários, nem exageros numa linguagem excessivamente técnica e
rebuscada. A conclusão deve dar respostas às questões do trabalho, correspondente aos
objetivos propostos. Deve ser breve, podendo, se necessário, apresentar sugestões para
pesquisas futuras.

\newpage

%%%%%%%% REFERÊNCIAS %%%%%%%%%%%%%%%%%


\bibliographystyle{bib/ppgee}
\bibliography{bib/bibliografia}

% Referências bibliográficas (geradas automaticamente)
\addcontentsline{toc}{chapter}{Referências bibliográficas}


\appendix

%Apêndice A
\include{apendice}

\end{document}

